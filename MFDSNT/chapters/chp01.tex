\newpage

\section{The Division Algorithm} 

\begin{axiom}{(The Well Ordering Principle)}
Every nonempty subset $S \subseteq \mathbb{N}$ has a smallest element.
\end{axiom}

\begin{theorem}{(Division Algorithm)}
If $a$ and $b$ are integers, with $b \geq 1$, then there exist unique integers $q$ and $r$, with $0 \le r < b$, such that $a = qb + r$. 
We call $q$ the quotient, and $r$ the remainder.
\end{theorem}

\begin{proof}
Let $a \in \mathbb{Z}, b \in \mathbb{N}.$ \\
Now, consider the set $S=\{z \in \mathbb{N} | z= a - bx \land x \in \mathbb{Z} \}$, this set is non-empty since\\
\hspace*{0.5 cm}if $a \geq 0 \land x=0$ then $z=a$, and\\
\hspace*{0.5 cm}if $a < 0 \land x=-2a$ then $z=a(1-2b)$ and positive ($\in \mathbb{N}$) since both factors are negative.\\
Therefore, by the Well Ordering Principle, $S$ has a minimal element, which we will call $r$.\\
Since $r \in S$, r is of the form $a-bx$, we will call $x=q$, so there are integers $r,q$ such that $r=a-bq$ or $a=qb+r$.\\
\newline
We still have to show that $r < b$. By contradiction we assume that $r \geq b$. Then, since $r \geq b$, clearly $r-b \geq 0$.
So $0 \leq (r-b) = (a-qb)-b=a-(q+1)b$. Since $a-(q+1)b \geq 0$ it is an element of $S$. Because $b>0$ we know that $r-b<r$.
We conclude that $r-b=a-(q+1)b < r$, but $r$ is the least element of $S$. Here we ran into a contradiction. Therefore
$r \geq b$ must be false and thus $r < b$.\\
\newline
We still have to show that both $q$ and $r$ are unique. Suppose we have $(q_1,r_1)$ and $(q_2,r_2)$ such that:\\
\hspace*{0.5 cm}$a = bq_1+r_1 = bq_2+r_2$, and\\
\hspace*{0.5 cm}$0 \le r_1 < b$, and\\
\hspace*{0.5 cm}$0 \le r_2 < b$.\\
We must show that necessarily $q_1=q_2$ and $r_1=r_2$.\\
Suppose, without loss of generality that $r_1 \geq r_2$, therefore $r_1-r_2 \geq 0$.\\
Since $a=q_1b+r_1=q_2b+r_2$, we know that $q_2b-q_1b=r_1-r_2$.\\
So
\hspace*{0.5 cm}$0 \leq r_1 -r_2$ \\
\hspace*{0.5 cm}$0 \leq (q_2 -q_1)b < b$, or \\
\hspace*{0.5 cm}$0 \leq (q_2 -q_1) < 1$.\\
Because $(q_2-q_1)$ must be an integer we know that $q_2-q_1=0$, and thus $q_1=q_2$ and also\\
$r_1=a-q_1 b= a - q_2 b = r_2$.\\
\newline
So we have found that there exist a unique $(q,r)$ such that $a=q b + r$ and $0 \leq r < b$. 
\end{proof}